\section{}





\addtocounter{subsection}{1}





\subsection{}

We want to clarify how the described gradings~$\Fcal_\bullet$ and~$\Fcal'_\bullet$ are supposed to be constructed:

For this let~$A$ be a~{\kalg} and let~$a_i \in A$ with~$i \in I$ be some elements of~$A$.
Then there exist a unique algebra homorphism~$f \colon k\gen{X_i \suchthat i \in I} \to A$ with~$f(X_i) = a_i$ for every~$i \in I$.
Suppose in the following that the~$a_i$ form a generating set for the algebra~$A$.
The algebra homomorphism~$f$ is then surjective.
If we are given for every~$i \in I$ a \enquote{degree}~$d_i$, i.e.\ a natural number~$d_i \geq 0$, then we can construct a filtration~$\Fcal_\bullet$ on~$A$ as follows:

\begin{itemize}
  \item
    There exist a unique grading on~$k\gen{X_i \suchthat i \in I}$ for which~$X_i$ has degree~$d_i$:
    We need to set
    \[
        k\gen{X_i \suchthat i \in I}_d
      \defined
        \gen{
          X_{i_1} \dotsm X_{i_r}
        \suchthat
          d_{i_1} + \dotsb + d_{i_r} = d
        }
    \]
    for every~$d \geq 0$.
    Then~$k\gen{X_i \suchthat i \in I} = \bigoplus_{d \geq 0} k\gen{X_i \suchthat i \in I}_d$, and it holds for all~$d, d' \geq 0$ that
    \[
        k\gen{X_i \suchthat i \in I}_d
        \cdot
        k\gen{X_i \suchthat i \in I}_{d'}
      \subseteq
        k\gen{X_i \suchthat i \in I}_{d+d'} \,.
    \]
  \item
    The above grading results in a filtration~$\tilde{\Fcal}_\bullet$ on~$k\gen{X_i \suchthat i \in I}$ given by
    \[
                \tilde{\Fcal}_d
      \defined  \bigoplus_{d' \leq d} k\gen{X_i \suchthat i \in I}_{d'} \,.
    \]
  \item
    By using the surjective algebra homomorphism~$f \colon k\gen{X_i \suchthat i \in I} \to A$ we can now define a filtration~$\Fcal_\bullet$ on~$A$ via
    \[
                \Fcal_d
      \defined  f( \tilde{\Fcal}_d ) \,.
    \]
\end{itemize}

The constructed filtration~$\Fcal_\bullet$ constructed in the above way satisfies
\[
      a_i
  =   f(X_i)
  \in f(\tilde{\Fcal}_{d_i})
  =   \Fcal_{d_i}
\]
for every~$i \in I$.

\begin{remark}
  The filtration~$\Fcal_\bullet$ is minimal with this property, i.e.\ if~$\Fcal'_\bullet$ is another filtration on~$A$ with~$a_i \in \Fcal'_{d_i}$ for every~$i \in I$, then~$\Fcal_d \subseteq \Fcal'_d$ for every~$d \geq 0$.
  Indeed, the linear subspace~$\tilde{\Fcal}_d$ of~$k\gen{X_i \suchthat i \in I}$ has the monomials
  \[
    X_{i_1} \dotsm X_{i_r}
    \qquad
    \text{with~$d_{i_1} + \dotsb + d_{i_r} \leq d$}
  \]
  as a basis.
  The linear subspace~$\Fcal_d = f(\tilde{\Fcal}_d)$ of~$A$ is therefore generated by the monomials
  \[
    a_{i_1} \dotsm a_{i_r}
    \qquad
    \text{with~$d_{i_1} + \dotsb + d_{i_r} \leq d$} \,.
  \]
  These monomials are by assumption contained in~$\Fcal'_d$, and hence~$\Fcal_d$ is contained in~$\Fcal'_d$.
\end{remark}

\begin{warning}
  It could still happen that~$a_i$ is of degree strictly smaller than~$d_i$ with respect to the filtration~$\Fcal$, i.e.\ it could happen that~$a_i \in \Fcal_d$ for some~$d < d_i$.
  
  Consider for example the case that~$A = k[x]$ with~$a_1 = x$ and~$a_2 = x^2$.
  If we choose~$d_1 = 1$ and~$d_2 = 3$ then the resulting filtration~$\Fcal$ will be given by
  \begin{align*}
        \Fcal_d
     =  \gen{
          a_1^{n_1} a_2^{n_2}
        \suchthat
          n_1 + 3 n_2 \leq d
        }
    &=  \gen{
          x^{n_1 + 2 n_2}
        \suchthat
          n_1 + 3 n_2 \leq d
        } \\
    &=  \gen{1, x, \dotsc, x^d}_k
     =  \{
          f \in k[x]
        \suchthat
          \deg(f) \leq d
        \}
  \end{align*}
  for every~$d \geq 0$.
  The filtration~$\Fcal$ is hence the standard filtration of~$k[x]$.
  From this we see that~$a_2 = x^2$ does not have degree~$d_2 = 3$, but instead degree~$2$.
\end{warning}

