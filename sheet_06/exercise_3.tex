\section{}

Recall the universal properties of the direct sum and the direct product:

\begin{itemize}
  \item
    Let~$(M_\alpha)_{\alpha \in A}$ be a collection of~{\modules{$R$}} and let~$N$ be another~{\module{$R$}}.
    For every~$\alpha \in A$ let~$i_\alpha \colon M_\alpha \to \bigoplus_{\beta \in A} M_\beta$ be the inclusion into the~\dash{$\alpha$}{th} summand.
    
    Then every choice of homomorphism~$f_\alpha \colon M_\alpha \to N$ with~$\alpha \in A$ can be uniquely extended to a homomorphism~$f \colon \bigoplus_{\alpha \in A} M_\alpha \to N$, in the sense that~$f \circ i_\alpha = f_\alpha$ for every~$\alpha \in A$.
    The homomorphism~$f$ is given on elements by
    \[
        f\left( (m_\alpha)_{\alpha \in A} \right)
      = \sum_{\alpha \in A} f_\alpha(m_\alpha)
    \]
    for every~$(m_\alpha)_{\alpha \in A} \in \bigoplus_{\alpha \in A} M_\alpha$.
    Note that~$f$ is~{\welldef} because~$m_\alpha = 0$ for all but finitely many~$\alpha \in A$, and hence also~$f_\alpha(m_\alpha) = 0$ for all but finitely many~$\alpha \in A$.
    
    This construction results in an isomorphism of abelian groups
    \[
        \Hom_R\left( \bigoplus_{\alpha \in A} M_\alpha, N \right)
      \longleftrightarrow
        \prod_{\alpha \in A} \Hom_R(M_\alpha, N) \,.
    \]
  \item
    Let~$M$ be an~{\module{$R$}} and let~$(N_\alpha)_{\alpha \in A}$ be a collection of~{\modules{$R$}}.
    For every~$\alpha \in A$ let~$p_\alpha \colon \prod_{\beta \in A} N_\beta \to N_\alpha$ be the projection onto the~\dash{$\alpha$}{th} factor.
    
    Then every choice of homomorphism~$f_\alpha \colon M \to N_\alpha$ with~$\alpha \in A$ can be uniquely combined into a homomorphism~$f \colon M \to \prod_{\alpha \in A} N_\alpha$, in the sense that~$p_\alpha \circ f = f_\alpha$ for every~$\alpha \in A$.
    The homomorphism~$f$ is given on elements by
    \[
        f(m)
      = ( f_\alpha(m) )_{\alpha \in A}
    \]
    for every~$m \in M$.
    
    This construction results in an isomorphism of abelian groups
    \begin{equation}
      \label{iso for product}
        \Hom_R\left( M, \prod_{\alpha \in A} M_\alpha \right)
      \longleftrightarrow
        \prod_{\alpha \in A} \Hom_R(M, N_\alpha) \,.
    \end{equation}
\end{itemize}

\begin{warning}
  The isomorphism~\eqref{iso for product} does in general not restrict to an isomorphism
  \[
      \Hom_R\left( M, \bigoplus_{\alpha \in A} M_\alpha \right)
    \longleftrightarrow
      \bigoplus_{\alpha \in A} \Hom_R(M, N_\alpha) \,.
  \]
  But it does if~$M$ is finitely generated.
\end{warning}

We now have that
\begin{align*}
        R'
  =     \Hom_\Integer(E, E)
  =     \Hom_\Integer\left( \bigoplus_p \Integer/p , \bigoplus_q \Integer/q \right)
  \cong \prod_p \Hom_\Integer \left( \Integer/p, \bigoplus_q \Integer/q \right)
\end{align*}
The inclusion~$i \colon \bigoplus_q \Integer/q \to \prod_q \Integer/q$ induces an inclusion of abelian groups
\[
                \Hom_\Integer \left( \Integer/p, \bigoplus_q \Integer/q \right)
  \xlongto{i_*} \Hom_\Integer \left( \Integer/p, \prod_q \Integer/q \right)
  \cong         \prod_q \Hom_\Integer( \Integer/p, \Integer/q ) \,.
\]
It follows from Schur’s lemma that~$\Hom_\Integer(\Integer/p, \Integer/q) = 0$ if~$p \neq q$.
(This can also be seen by looking at the~\dash{$p$}{torsion} (or~\dash{$q$}{torsion}) of both sides.)
We have for~$p = q$ that
\[
        \Hom_\Integer( \Integer/p, \Integer/p )
  =     \Hom_{\Integer/p}( \Integer/p, \Integer/p )
  =     \End_{\Integer/p}( \Integer/p )
  \cong \Integer/p \,.
\]
Hence
\[
        \Hom_\Integer \left( \Integer/p, \prod_q \Integer/q \right)
  \cong \Integer/p \,,
\]
where for~$\lambda \in \Integer/p$ the corresponding homomorphism is given by
\[
  \Integer/p
  \xlongto{\lambda \cdot (-)}
  \Integer/p
  \inclusion
  \prod_q \Integer/q \,.
\]
Every such homomorphism restricts to a homomorphism~$\Integer/p \to \bigoplus_q \Integer/q$, hence the above homomorphism~$i_*$ is already an isomorphism.
We thus find that
\[
        \Hom_\Integer \left( \Integer/p, \bigoplus_q \Integer/q \right)
  \cong \Integer/p \,,
\]
with the same description as above.

It follows that
\[
        R'
  \cong \prod_p \Hom_\Integer \left( \Integer/p, \bigoplus_q \Integer/q \right)
  \cong \prod_p \Integer/p \,,
\]
where an element~$(\lambda_p)_p \in \prod_p \Integer/p$ acts on~$(x_p)_p \in E$ via
\[
    (\lambda_p)_p \cdot (x_p)_p
  = (\lambda_p x_p)_p \,.
\]

To determine~$R''$ we use the following observation:

\begin{lemma}
  Let~$R$ be a commutative ring and let~$M$ be an~{\module{$R$}}.
  If~$R' = \End_R(M)$ is again commutative then~$R'' = R'$.
\end{lemma}

\begin{proof}
  It follows from~$R$ being commutative that~$R \subseteq R'$, and hence that~$R' \supseteq R''$.
  It holds that~$R' \subseteq R''$ because~$R'$ is commutative.
\end{proof}

We find that~$R'' = R' = \prod_p \Integer/p$.
The (unique) ring homomorphism
\[
      \Integer
  =   R
  \to R''
  =   \prod_p \Integer/p
\]
is not surjective, so~$R'' \neq R$.



