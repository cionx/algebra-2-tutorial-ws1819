\section{}

For every multiindex~$\mu = (\mu_1, \dotsc, \mu_n)$ we denote by~$x^\mu \in k[x_1, \dotsc, x_n]$ the element
\[
            x^\mu
  \defined  x_1^{\mu_1} \dotsm x_n^{\mu_n} \,.
\]





\subsection{}

For every~$f \in k[x_1, \dotsc, x_n]$ the element
\[
            f'
  \defined  \sum_{g \in G} g.f
\]
is~\dash{$G$}{invariant} because
\[
    g.f'
  = g.\sum_{g' \in G} g'.f
  = \sum_{g' \in G} g g'.f
  = \sum_{g'' \in G} g''.f
  = f \,.
\]
For every multiindex~$\mu = (\mu_1, \dotsc, \mu_n)$ we choose~$f = x^\mu$, and thus set
\[
            J_\mu
  \defined  \sum_{g \in G} g.x^\mu  \,.
\]





\subsection{}

It holds for every~\dash{$G$}{invariant}~$f \in k[x_1, \dotsc, x_n]^G$ that
\[
    \sum_{g \in G} g.f
  = \sum_{g \in G} f
  = \card{G} f \,.
\]
But with~$f = \sum_{\mu} a_\mu x^\mu$ it also holds that
\[
    \sum_{g \in G} g.f
  = \sum_{g \in G} g.\sum_{\mu} a_\mu x^\mu
  = \sum_{\mu} a_\mu \sum_{g \in G} g.x^\mu
  = \sum_{\mu} a_\mu J_\mu \,.
\]

\begin{remark}
  If~$\ringchar(k) \nmid \card{G}$ and~$V$ is any representation of~$G$ over~$k$ then the map
  \[
            R
    \colon  V
    \to     V \,,
    \quad   v
    \mapsto \frac{1}{\card{G}} \sum_{g \in G} g.v
  \]
  is a projection of~$V$ onto the subspace of invariants~$V^G \subseteq V$.
  This projection is known as the \emph{Reynolds operator}.
\end{remark}





\subsection{}

Let~$h \defined \card{G}$ and let~$G = \{g_1, \dotsc, g_h\}$.

It follows from the previous part of the exercise that~$k[x_1, \dotsc, x_n]$ is generated by the~\dash{$G$}{invariants}~$J_\mu$ as a vector space, where~$\mu \in \Natural^n$, because the factor~$\card{G}$ is invertible in~$k$.
It therefore sufficies to show that every~$J_\mu$ can be written as polynomial in those~$J_\nu$ for which~$\size{\nu} \leq h$.
  
For every~$j \geq 0$ let~$p_j = Y_1^j + \dotsb + Y_h^j \in k[Y_1, \dotsc, Y_h]$ be the~\dash{$j$}{th} power symmetric polynomial.
For the elements
\[
            y_i
  \defined  (g_i.x_1) Z_1 + \dotsb + (g_i.x_n) Z_n
  \in       k[x_1, \dotsc, x_n][Z_1, \dotsc, Z_n]
\]
with~$i = 1, \dotsc, h$ we then have that
\begingroup
\allowdisplaybreaks
\begin{align*}
      p_j(y_1, \dotsc, y_n)
  &=  y_1^j + \dotsb + y_h^j  \\
  &=  \sum_{i=1}^h \bigl[ (g_i.x_1) Z_1 + \dotsb + (g_i.x_n) Z_n \bigr]^j  \\
  &=  \sum_{i=1}^h \sum_{|\mu| = j}
      \binom{j}{\mu_1, \dotsc, \mu_n} [(g_i.x_1) Z_1]^{\mu_1} \dotsm [(g_i.x_n) Z_n]^{\mu_n}  \\
  &=  \sum_{i=1}^h \sum_{|\mu| = j}
      \binom{j}{\mu_1, \dotsc, \mu_n} (g_i.x_1)^{\mu_1} \dotsm (g_i.x_n)^{\mu_n} Z^\mu \\
  &=  \sum_{|\mu| = j} \binom{j}{\mu_1, \dotsc, \mu_n}
      \left[
        \sum_{i=1}^h (g_i.x_1)^{\mu_1} \dotsm (g_i.x_n)^{\mu_n}
      \right]
      Z^\mu  \\
  &=  \sum_{|\mu| = j} \binom{j}{\mu_1, \dotsc, \mu_n}
      \left[
        \sum_{i=1}^h g_i.x^\mu
      \right]
      Z^\mu  \\
  &=  \sum_{|\mu| = j} \binom{j}{\mu_1, \dotsc, \mu_n} J_\mu Z^\mu \,.
\end{align*}
\endgroup
This shows that~$J_\mu$ is, up to the factor
\[
            C_\mu
  \defined  \binom{|\mu|}{\mu_1, \dotsc, \mu_n} \,,
\]
the coefficient of the monomial~$Z^\mu$ in~$p_j(y_1, \dotsc, y_n)$.

We know that for every~$j > h$ the~\dash{$j$}{th} power symmetric polynomial~$p_j$ can be expressed as a~\dash{$k$}{polynomial} in the power symmetric polynomials~$p_1, \dotsc, p_h$.
It follows that the coefficients of~$p_j(y_1, \dotsc, y_n)$ are~\dash{$k$}{polynomial}s in the coefficients of~$p_1(y_1, \dotsc, y_n), \dotsc, p_h(y_1, \dotsc, y_n)$.

This shows that for every multiindex~$\mu$ the~\dash{$G$}{invariant}~$C_\mu J_\mu$ can be expressed as a~\dash{$k$}{polynomial} in the terms~$C_\nu J_\nu$ with~$|\nu| \leq h$.
The factor~$C_\mu$ is invertible in~$k$, hence every~$J_\mu$ is a~\dash{$k$}{polynomial} in those~$J_\nu$ with~$|\nu| \leq h$.


\begin{remark}
  Let~$V$ be a {\fd} of a finite group~$G$ over~$k$.
  The \emph{Noether number of~$V$} is given by
  \[
    \beta(V,G)
      =
    \inf
    \{
      d \geq 0
    \suchthat
      \text{$\Pol(V)^G$ is generated by homogeneous elements of degree~$\leq d$}
    \} \,,
  \]
  and the \emph{Noether number of~$G$} is given by
  \[
    \beta(G)
      \defined
    \sup
    \{
      \beta(V,G)
    \suchthat
      \text{$V$ is a {\fd} representation of~$G$ over~$k$}
    \} \,.
  \]
  Noether’s theorem (1915) shows that~$\beta(G) \leq |G|$ if~$\ringchar(k) = 0$, which is known as the \emph{Noether bound}.
  This result can be strengthened in various ways:
  \begin{itemize}
    \item
      Fogarty (2001) showed that that the Noether bound holds under the weaker assumption that~$\ringchar(k) \nmid \card{G}$.
    \item
      Fleischmann (2000) showed the more general result that if~$H \subseteq G$ is a normal subgroup with~$\ringchar(k) \nmid [G : H]$, then~$\beta(V,G) \leq \beta(V,H) \cdot [G : H]$.
    \item
      For~$\ringchar(k) = 0$ it was proven by Schmid (1991) that~$\beta(G) \leq \beta(H)[G : H]$ for every subgroup~$H \subseteq G$, and that~$\beta(G) \leq \beta(H)\beta(G/H)$ if~$H$ is normal in~$G$.
    \item
      It is an open problem if~$\beta(G) \leq \beta(H)[G : H]$ holds for every subgroup~$H \subseteq G$ under the weaker condition that~$\ringchar(k) \nmid [G : H]$.
  \end{itemize}
\end{remark}




