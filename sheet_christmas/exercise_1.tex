\section{}

We observe first that instead of~$1 \leq i, j \leq n$ the condition~$1 \leq i < j \leq n$ is needed.
We hence set
\[
  V
  \defined
  \prod_{1 \leq i < j \leq n} (X_i - X_j)
\]
and need to show that~$V \divides f$.

We consider first the case~$i = 1 < 2 = j$:
We note that
\begin{align*}
  k[X_1, \dotsc, X_n]/(X_1 - X_2)
  &\cong
  k[Y_1, \dotsc, Y_{n-1}] \,,
  \\
  \class{ g(X_1, \dotsc, X_n) }
  &\mapsto
  g(Y_1, Y_1, Y_2, \dotsc, Y_{n-1}) \,.
\end{align*}
We find that~$f$ is mapped to
\[
  f(Y_1, Y_1, Y_2, \dotsc, Y_{n-1})
  =
  -f(Y_1, Y_1, Y_2, \dotsc, Y_{n-1})  \,.
\]
It follows from~$\ringchar(k) \neq 2$ that~$f$ is mapped to~$0$, and hence that~$f \in (X_1 - X_2)$.
This shows that~$(X_1 - X_2) \divides f$.

We find in the same way that~$X_i - X_j \divides f$ for all~$1 \leq i < j \leq n$.
These elements are pairwise \dash{non}{equivalent} primes of~$k[X_1, \dotsc, X_n]$.
that they are prime follows from~$k[X_1, \dotsc, X_n]/(X_i - X_j) \cong k[Y_1, \dotsc, Y_{n-1}]$ being an integral domain, and they are \dash{non}{equivalent} because they are monic and pairwise different.
It follows, because~$k[X_1, \dotsc, X_n]$ is a UFD, that~$V \divides f$.


