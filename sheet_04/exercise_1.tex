\section{}





\subsection{}

The main problem with this exercise is that it can be understood in two ways:
\begin{enumerate}
  \item
    \label{polynomial map}
    We need show that a polynomial map~$f \colon W \to V$ is~\dash{$G$}{equivariant} if and only if the induced algebra homomorphism~$f^* \colon \Pol(V) \to \Pol(W)$ is~\dash{$G$}{invariant}.
  \item
    \label{any map}
    We need to show that an arbitrary map~$f \colon W \to V$ is covariant if and only if~$f^* \colon \Pol(V) \to \Pol(W)$,~$h \mapsto h \circ f$ is a~{\welldef} algebra homomorphism which is~\dash{$G$}{invariant}.
\end{enumerate}


We start by showing that~\ref*{polynomial map} and~\ref*{any map} are actually equivalent.
For this we need to show that a map~$f \colon W \to V$ is polynomial if and only if~$f^* \colon \Pol(V) \to \Pol(W)$,~$h \mapsto h \circ f$ is a~{\welldef} algebra homomorphisms.
We have already seen in the lecture that~$f^*$ is a {\welldef} algebra homomorphism if~$f$ is polynomial.

Let on the other hand~$f^*$ be a {\welldef} algebra homorphism.
Let~$v_1, \dotsc, v_n$ be a basis of~$V$.
Then there exist unique functions~$f_1, \dotsc, f_n \colon V \to k$ with
\[
    f(v)
  = f_1(v) w_1 + \dotsb + f_n(v) w_n
\]
for every~$v \in V$.
If~$\varphi_1, \dotsc, \varphi_n \in \Pol(V)$ denote the coordinate functions with respect to the basis~$v_1, \dotsc, v_n$ then
\[
    f_i(v)
  = \varphi_i( f(v) )
  = f^*(\varphi_i)(v)
\]
for every~$v \in V$ and hence
\[
      f_i
  =   f^*(\varphi_i)
  \in \Pol(V) \,.
\]
This shows that~$f_1, \dotsc, f_n$ are polynomial, and hence that~$f$ is polynomial.

We now show~\ref*{polynomial map}.
For this we calculate for all~$g \in G$,~$h \in \Pol(V)$ and~$w \in W$ that
\begin{align*}
   {}&  (g.f^*)(h)(w) \\
  ={}&  (g.f^*(g^{-1}.h))(w)  \\
  ={}&  f^*(g^{-1}.h)(g^{-1}.w) \\
  ={}&  ((g^{-1}.h) \circ f)(g^{-1}.w)  \\
  ={}&  (g^{-1}.h)( f( g^{-1}.w ) ) \\
  ={}&  h( g.f( g^{-1}.w ) ) \,.
\end{align*}

We also note that for every two elements~$v, v' \in V$ with~$v \neq v'$ there exist some polynomial function~$h \in \Pol(V)$ with~$h(v) \neq h(v')$.
Indeed, if we choose again a basis~$v_1, \dotsc, v_n$ of~$V$ then it holds for~$v = \sum_{i=1}^n a_i v_i$ and~$v' = \sum_{i=1}^n a'_i v_i$ with~$a_i, a'_i \in k$ that~$a_i \neq a'_i$ for some~$i$.
For the coordinate functions~$\varphi_1, \dotsc, \varphi_n \in \Pol(V)$ with respect to the basis~$v_1, \dotsc, v_n$ we therefore have that~$\varphi_i(v) = a_i \neq a'_i = \varphi_i(v')$.
So we may choose~$h = \varphi_i$.

We therefore find that
\begin{align*}
   {}&  \text{$f^*$ is~\dash{$G$}{invariant}} \\
  ={}&  g.f^* = f^* \\
  ={}&  \text{$(g.f^*)(h)(w) = f^*(h)(w)$ for all~$h \in \Pol(V)$,~$w \in W$} \\
  ={}&  \text{$h(g.f(g^{-1}.w)) = h(f(w))$ for all~$h \in \Pol(V)$,~$w \in W$}  \\
  ={}&  \text{$g.f(g^{-1}.w) = f(w)$ for all~$w \in W$} \\
  ={}&  \text{$(g.f)(w) = f(w)$ for all~$w \in W$}  \\
  ={}&  g.f = f \\
  ={}&  \text{$f$ is~\dash{$G$}{invariant}} \\
  ={}&  \text{$f$ is~\dash{$G$}{equivariant}} \,,
\end{align*}
where we use again (as seen in the lecture and in the tutorial) that a map~$f \colon W \to V$ is~\dash{$G$}{invariant} if and only if it is~\dash{$G$}{equivariant}.





\subsection{}

We know from the lecture that~$\Pol(\mat{n}{k})^{\SL_n(k)} = k[\det]$, hence that~$\Pol(\mat{n}{k})^{\SL_n(k)}$ has as a basis the determinant powers~$1, \det, \det^2, \dotsc$
The grading of~$\Pol(\mat{n}{k})^{\SL_n(k)}$ is inherited from~$\Pol(\mat{n}{k})$, and the determinant~$\det$ is homogeneous of degree~$d$ by the Leibniz formula
\[
    \det
  = \sum_{\sigma \in S_n} \sign(\sigma) x_{1\sigma(1)} \dotsm x_{n\sigma(n)} \,.
\]
We thus find that the determinant power~$\det^m$ has homogeneous degree~$nm$ for all~$m \geq 0$.
We can now pull back the grading of~$\Pol(\mat{n}{k})^{\SL_n(k)}$ via the isomorphism
\[
          k[t]
  \to     \Pol(\mat{n}{k})^{\SL_n(k)}
  \quad   p(t)
  \mapsto p(\det)
\]
to a grading of~$k[t]$;
the basis element~$\det^m$ of~$\Pol(\mat{n}{k})^{\SL_n(k)}$ corresponds to the basis element~$t^m$ of~$k[t]$, which is therefore of homogeneous degree~$nm$.
This shows that
\[
    k[t]_{mn}
  = \gen{ t^m }_k
\]
for all~$m \geq 0$, and~$k[t]_d = 0$ otherwise.
It follows in particular that
\[
    \dim k[t]_{mn}
  = 1
\]
for all~$m \geq 0$ and~$\dim k[t]_d = 0$ otherwise.
The Hilbert series of~$k[t]$ (with the above grading) is therefore given by
\[
    \sum_{m \geq 0} t^{nm}
  = \sum_{m \geq 0} (t^n)^m
  = \frac{1}{1-t^n} \,.
\]
